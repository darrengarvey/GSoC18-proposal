%
% (c) Tom Westerhout, 2018
%

\documentclass[a4paper,12pt]{article}

\usepackage[margin=2cm]{geometry}
\usepackage[utf8]{inputenc}
\usepackage[english]{babel}
\usepackage{textcomp}

\usepackage{amsmath}
\usepackage{amsfonts}
\usepackage{color}
\usepackage{verbatim}

\usepackage{setspace}
\usepackage{scrextend}
\usepackage{multirow}
\usepackage{pbox}

\usepackage{enumitem}
\usepackage{tikz}
\usepackage{csquotes}
\usepackage{hyperref}

\usepackage{listings}

\lstset{ %
    backgroundcolor=\color{white},   % choose the background color; you must add \usepackage{color} or \usepackage{xcolor}; should come as last argument
    basicstyle=\scriptsize\ttfamily,        % the size of the fonts that are used for the code
    breakatwhitespace=false,         % sets if automatic breaks should only happen at whitespace
    breaklines=true,                 % sets automatic line breaking
    captionpos=b,                    % sets the caption-position to bottom
    commentstyle=\itshape\color[HTML]{555555},    % comment style
    %deletekeywords={...},            % if you want to delete keywords from the given language
    %escapeinside={\%*}{*)},          % if you want to add LaTeX within your code
    extendedchars=true,              % lets you use non-ASCII characters; for 8-bits encodings only, does not work with UTF-8
    frame=single,	                   % adds a frame around the code
    keepspaces=true,                 % keeps spaces in text, useful for keeping indentation of code (possibly needs columns=flexible)
    keywordstyle=\bfseries\color[HTML]{2D578C},       % keyword style
    language=C++,                 % the language of the code
    morekeywords={constexpr,noexcept},           % if you want to add more keywords to the set
    numbers=left,                    % where to put the line-numbers; possible values are (none, left, right)
    numbersep=10pt,                   % how far the line-numbers are from the code
    numberstyle=\tiny\color[HTML]{333333}, % the style that is used for the line-numbers
    rulecolor=\color{black},         % if not set, the frame-color may be changed on line-breaks within not-black text (e.g. comments (green here))
    showspaces=false,                % show spaces everywhere adding particular underscores; it overrides 'showstringspaces'
    showstringspaces=false,          % underline spaces within strings only
    showtabs=false,                  % show tabs within strings adding particular underscores
    stepnumber=2,                    % the step between two line-numbers. If it's 1, each line will be numbered
    stringstyle=\color[HTML]{488A4C},     % string literal style
    tabsize=2,	                   % sets default tabsize to 2 spaces
    title=\lstname                   % show the filename of files included with \lstinputlisting; also try caption instead of title
}


\title{Google Summer of Code Proposal: Boost.StaticViews}
\date{March 2018}

\bibliography{bibliography}


\begin{document}
\maketitle

\begin{abstract}
    \textbf{Yet to be written.} For now, here's the last year's abstract:

    This document proposes an addition to \texttt{Boost C++ Libraries} -- a
    \textit{compile-time} hash table. There are multiple good implementations of
    unordered associate containers (e.g. \texttt{std::unordered\_map},
    \texttt{Google}'s \texttt{sparsehash}). These implementations provide both
    lookup and insertion/deletion functionality. They are, however, not the
    perfect fit for the case when the contents of the container are fixed upon
    construction or even known at compile time. \texttt{std::vector} vs.
    \texttt{std::array} is a good analogy here. We propose a
    \texttt{static\_map} -- an associate container with focus on
    \texttt{constexpr} usage.
\end{abstract}

\section{Personal Details}
    \begin{labeling}{Degree Program}
    \item [Name] Tom Westerhout
    \item [University] Radboud University of Nijmegen, The Netherlands
    \item [Degree Program] Masters in Physics
    \item [Email] \texttt{kot.tom97@gmail.com}
    \item [Availability] I am available from May till the end of August. I
        consider \texttt{GSoC} a full-time job/internship and thus plan to spend
        at least 40 hours per week working on the project. 14th (Monday) of May
        (the official start date) seems like a good starting point to me. My
        summer vacation officially starts only in the beginning of July, before
        that I have some courses to follow, exams to take and a research project
        to work on. Although it seems much, last year's experience shows that I
        still find enough time to work on the project. Also, I will have more
        free time in July and August and can easily compensate for the lost
        time, if any.
    \end{labeling}

\section{Background Information}
    I am currently in my first year of Masters in Physics program at Radboud
    University (The Netherlands). Although I am a Physics major, I am also
    passionate about programming. I follow many programming-related courses
    and am considering doing a Masters in Computer Science. So far, I have
    followed the following project-related courses:
    \begin{itemize}
        \item \textit{Imperative Programming 1 \& 2} (Introduction to \texttt{C++}).
        \item \textit{Hacking in C} (Memory layout, calling conventions and
            debugging).
        \item \textit{Object Orientation}.
        \item \textit{Algorithms \& Datastructures} (I've also worked as a
            TA (Teaching Assistant) for this course).
        \item \textit{Languages and Automata}.
        \item \textit{Functional Programming 1 \& 2}.
        \item \textit{Advanced Programming} (Generic programming, DSLs (Domain
            Specific Languages), etc.)
        \item \textit{Compiler Construction}.
    \end{itemize}

    Also, in January 2018 I have participated in 3COWS (Composability,
    Comprehensibility, Correctness Winter School). It was an intensive programme
    for Ms. and PhD. students extending the CEFP (Central European Functional
    Programming) summer school.

    For two and a half years I have been worked as a teaching assistant at
    Radboud University. I have taught physics, math, and computer science
    courses.

    As for the projects, as part of my Bachelor internship, I have written some
    \texttt{C++14} code for distributed memory systems for doing some physical
    calculations. We have recently published a paper about the results of these
    computations. Although the algorithms involved are not particularly
    difficult, as far as I know, this is still the only open source solution for
    calculating plasmonic properties of non-translational invariant quantum
    systems.

    Currently, I am working on applying swarm intelligence to obtaining excited
    states of many-body quantum systems. High level algorithms are written in
    \texttt{Haskell} and performance critical parts --- in \texttt{C++17} with a
    \texttt{C} wrapper in between. We use shallow neural network (Restricted
    Boltzmann Machine) as a variational ansatz by tracing out all hidden nodes.
    This part is written in \texttt{C++}. Also, Monte-Carlo simulations which
    are used to estimate the fitness function are written in \texttt{C++} as
    well. We hope to present some first results at ``Beyond Digital Computing''
    conference in Heidelberg near end of March 2018.

    Finally, last summer I have successfully completed a Google Summer of Code
    working on the very same project I am proposing here. I believe StaticViews
    library to be very useful. It is however not possible to create a
    Boost-quality library in just one summer. Having a relatively high study
    load (due to my following both Physics and CS courses) I don't have enough
    time to work on the project from September till May. I did however manage to
    follow all the major discussions on \texttt{boost-dev} mailing list. Also, I
    feel like I may need some guidance to complete this project. These two
    reasons are my main motivation for doing GSoC this year.

    As for my knowledge of the listed languages/technologies/tools, I would rate
    it as follows:
    \begin{labeling}{\texttt{C++ Standard Library}}
    \item [\texttt{C++ 98/03}] $\approx$ 3;
    \item [\texttt{C++ 11/14/17}] $\approx$ 4;
    \item [\texttt{C++ Standard Library}] $\approx$ 4;
    \item [\texttt{Boost C++ Libraries}] $\approx$ 4;
    \item [\texttt{Git}] $\approx$ 3.
    \end{labeling}

    I use \texttt{Linux} as my primary desktop operating system and have grown
    used to doing as much as possible via command line. Thus, usually I write
    code in \texttt{Neovim} and use some build system to compile it. From the
    \texttt{C++}--related ones I am familiar with \texttt{Boost.Build} and
    \texttt{CMake}. For documentation, I have grown to like \texttt{Sphinx}
    (which StaticViews currently uses). I also have experience in using
    \texttt{Doxygen} and have played around with \texttt{Standardese}.

\section{Project Proposal}
    \textbf{Yet to be written.}

\end{document}
