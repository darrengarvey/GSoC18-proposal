%
% (c) Tom Westerhout, 2018
%

\documentclass[a4paper,12pt]{article}

\usepackage[margin=2cm]{geometry}
\usepackage[utf8]{inputenc}
\usepackage[english]{babel}
\usepackage{textcomp}

\usepackage{amsmath}
\usepackage{amsfonts}
\usepackage{color}
\usepackage{verbatim}

\usepackage{setspace}
\usepackage{scrextend}
\usepackage{multirow}
\usepackage{pbox}

\usepackage{enumitem}
\usepackage{tikz}
\usepackage{csquotes}
\usepackage[colorlinks=false]{hyperref}

\hypersetup{
    colorlinks = true,
    urlcolor = [HTML]{2D578C},
    citecolor = [HTML]{555555},
    linkcolor = [HTML]{555555}
}

\usepackage{listings}

\lstset{ %
    backgroundcolor=\color{white},   % choose the background color; you must add \usepackage{color} or \usepackage{xcolor}; should come as last argument
    basicstyle=\scriptsize\ttfamily,        % the size of the fonts that are used for the code
    breakatwhitespace=false,         % sets if automatic breaks should only happen at whitespace
    breaklines=true,                 % sets automatic line breaking
    captionpos=b,                    % sets the caption-position to bottom
    commentstyle=\itshape\color[HTML]{555555},    % comment style
    %deletekeywords={...},            % if you want to delete keywords from the given language
    %escapeinside={\%*}{*)},          % if you want to add LaTeX within your code
    extendedchars=true,              % lets you use non-ASCII characters; for 8-bits encodings only, does not work with UTF-8
    frame=single,	                   % adds a frame around the code
    keepspaces=true,                 % keeps spaces in text, useful for keeping indentation of code (possibly needs columns=flexible)
    keywordstyle=\bfseries\color[HTML]{2D578C},       % keyword style
    language=C++,                 % the language of the code
    morekeywords={constexpr,noexcept},           % if you want to add more keywords to the set
    numbers=left,                    % where to put the line-numbers; possible values are (none, left, right)
    numbersep=10pt,                   % how far the line-numbers are from the code
    numberstyle=\tiny\color[HTML]{333333}, % the style that is used for the line-numbers
    rulecolor=\color{black},         % if not set, the frame-color may be changed on line-breaks within not-black text (e.g. comments (green here))
    showspaces=false,                % show spaces everywhere adding particular underscores; it overrides 'showstringspaces'
    showstringspaces=false,          % underline spaces within strings only
    showtabs=false,                  % show tabs within strings adding particular underscores
    stepnumber=2,                    % the step between two line-numbers. If it's 1, each line will be numbered
    stringstyle=\color[HTML]{488A4C},     % string literal style
    tabsize=2,	                   % sets default tabsize to 2 spaces
    title=\lstname                   % show the filename of files included with \lstinputlisting; also try caption instead of title
}

\setlist{nosep}

\title{Google Summer of Code Proposal: Boost.StaticViews}
\date{March 2018}

\bibliographystyle{plain}

\begin{document}
\maketitle

\begin{abstract}
    This document proposes an addition to Boost C++ Libraries ---
    \texttt{StaticViews} library. The library focuses on working with
    compile-time (i.e.  \texttt{constexpr}) homogeneous data. Applications range
    from converting bitmaps from 8-bit to 24-bit representations to implementing
    efficient enumeration to string conversions and custom error categories.
\end{abstract}

\vspace{5cm}
\section*{Personal Details}
    \begin{labeling}{Degree Program}
    \item [Name] Tom Westerhout
    \item [University] Radboud University of Nijmegen, The Netherlands
    \item [Degree Program] Masters in Physics
    \item [Email] \texttt{kot.tom97@gmail.com}
    \item [Availability] I am available from May till the end of August. I
        consider GSoC a full-time job/internship and thus plan to spend at least
        40 hours per week working on the project. 14th (Monday) of May (the
        official start date) seems like a good starting point to me. My summer
        vacation officially starts only in the beginning of July, before that I
        have some courses to follow, exams to take and a research project to
        work on. Although it seems much, last year's experience shows that I
        still find enough time to work on the project. Also, I will have more
        free time in July and August and can easily compensate for the lost
        time, if any. I might also decide to keep working full-time on the
        project till the end of August rather than the official end of GSoC.
    \end{labeling}

\newpage
\section*{Background Information}
    I am currently in my first year of Masters in Physics program at Radboud
    University (in the Netherlands). Although I am a Physics major, I am also
    passionate about programming. I follow many programming-related courses
    and am considering doing a Masters in Computer Science. So far, I have
    followed the following project-related courses:
    \begin{itemize}
        \item \textit{Imperative Programming 1 \& 2} (Introduction to \texttt{C++}).
        \item \textit{Hacking in C} (Memory layout, calling conventions and
            debugging).
        \item \textit{Object Orientation}.
        \item \textit{Algorithms \& Datastructures} (Also worked as a
            TA (Teaching Assistant) for this course).
        \item \textit{Languages and Automata}.
        \item \textit{Functional Programming 1 \& 2}.
        \item \textit{Advanced Programming} (Generic programming, DSLs (Domain
            Specific Languages), etc.)
        \item \textit{Compiler Construction}.
    \end{itemize}

    Also, in January 2018 I have participated in 3COWS (Composability,
    Comprehensibility, Correctness Winter School). It was an intensive programme
    for Ms. and PhD. students extending the CEFP (Central European Functional
    Programming) summer school.

    For two and a half years I have been worked as a teaching assistant at
    Radboud University. I have taught physics, math, and computer science
    courses.

    As for the projects, as part of my Bachelor internship\cite{bachelor2017}, I
    have written some \texttt{C++14} code\cite{plasmon-cpp} for distributed
    memory systems for doing some physical calculations. We have recently
    published a paper\cite{westerhout2018plasmon} about the results of these
    computations. Although the algorithms involved are not particularly
    difficult, as far as I know, this is still the only open source solution for
    calculating plasmonic properties of non-translational invariant quantum
    systems.

    Currently, I am working on applying swarm intelligence to obtaining excited
    states of many-body quantum systems\cite{tcm-swarm}. High level algorithms
    are written in \texttt{Haskell} and performance critical parts --- in
    \texttt{C++17} with a \texttt{C} wrapper in between. We use shallow neural
    network (Restricted Boltzmann Machine) as a variational ansatz by tracing
    out all hidden nodes. This part is written in \texttt{C++}. Monte-Carlo
    simulations which are used to estimate the fitness function are written in
    \texttt{C++} as well. We hope to present some first results at ``Beyond
    Digital Computing'' conference in Heidelberg near end of March 2018.

    Finally, last summer I have successfully completed a Google Summer of Code
    working on the very same project I am proposing here. I believe StaticViews
    library to be very useful. It is however not possible to create a
    Boost-quality library in just one summer. Having a relatively high study
    load (due to my following both Physics and CS courses) I don't have enough
    time to work on the project from September till May. I did however manage to
    follow all the major discussions on \texttt{boost-dev} mailing list. Also, I
    feel like I may need some guidance to complete this project. These two
    reasons are my main motivation for doing GSoC this year.

    As for my knowledge of the listed languages/technologies/tools, I would rate
    it as follows:

    \begin{tabular}{l r}
        \texttt{C++ 98/03} & $\approx$ 3; \\
        \texttt{C++ 11/14/17} & $\approx$ 4; \\
        \texttt{C++ Standard Library} & $\approx$ 4; \\
        \texttt{Boost C++ Libraries} & $\approx$ 4; \\
        \texttt{Git} & $\approx$ 3.
    \end{tabular}

    I use \texttt{Linux} as my primary desktop operating system and have grown
    used to doing as much as possible via command line. Thus, usually I write
    code in \texttt{Neovim} and use some build system to compile it. From the
    \texttt{C++}--related ones I am familiar with \texttt{Boost.Build} and
    \texttt{CMake}. For documentation, I have grown to like \texttt{Sphinx}
    (which StaticViews currently uses). I also have experience in using
    \texttt{Doxygen} and have played around with \texttt{Standardese}.

\newpage
\section*{Project Proposal}
    \paragraph{Ranges TS, Boost.Hana and StaticViews} Computations in
    \texttt{C++} can be divided into \textit{run-time} and \textit{compile-time}
    ones.\footnote{%
        Without loss of generality, we can say that all computations produce
        results, because if a computation results in nothing (\texttt{void}), we
        can always create a wrapper around it returning a dummy value. Then we
        can easily distinguish run- and compile-time computations by defining a
        computation as compile-time if its result can be stored in a
        \texttt{constexpr} variable.%
    }
    If we talk about run-time manipulation of homogeneous sequences, then quite
    often STL algorithms\cite{stl-algorithms} are enough. If one wants to do
    multiple things in one pass, Ranges TS\cite{ranges-ts} come into play. For
    run-time manipulation of heterogeneous sequences there is
    Boost.Fusion\cite{boost-fusion}. There is also Boost.Hana\cite{boost-hana}
    which greatly simplifies working with compile-time heterogeneous sequences.
    But what about homogeneous compile-time data? One could naively expect that
    Boost.Hana can handle that as well. It turns out to be not the case.

    Hana focuses on type-level information. This means that if one wants to, for
    example, find all floats in a constexpr \texttt{C}-style array which are
    smaller than some threshold, one can't do that in Hana in a non-hacky way.
    This example illustrates that there is a gap between run-time computations
    (Ranges TS) and type-level computations (Boost.Hana). StaticViews is meant
    to fill this gap. The library consists of two layers: low-level views and
    high-level data data structures implemented on top of them.

    \paragraph{Static Map} \texttt{static\_views::static\_map} is an example
    (currently, the only one) of such high-level data structures. Simply put, it
    is a compile-time analogue of \texttt{std::unordered\_map}. Consider the
    following toy example:

\begin{spacing}{0.8}
\begin{lstlisting}
enum class weekday {
  sunday, monday, tuesday, wednesday, thursday, friday, saturday};

#define STRING_VIEW(str) string_view{str, sizeof(str)-1}
constexpr static std::pair<string_view const, weekday>
  string_to_weekday[] =
    {{ STRING_VIEW("sunday"),    weekday::sunday    },
     { STRING_VIEW("monday"),    weekday::monday    },
     { STRING_VIEW("tuesday"),   weekday::tuesday   },
     { STRING_VIEW("wednesday"), weekday::wednesday },
     { STRING_VIEW("thursday"),  weekday::thursday  },
     { STRING_VIEW("friday"),    weekday::friday    },
     { STRING_VIEW("saturday"),  weekday::saturday  }};

int main()
{
  {
    // Calls malloc() at least 7 times
    static const std::map<string_view, weekday> to_weekday1{
      std::begin(string_to_weekday), std::end(string_to_weekday)};
    std::cout << "'monday' maps to "
              << static_cast<int>(to_weekday1.at("monday"))
              << std::endl;
    std::cout << "'friday' maps to "
              << static_cast<int>(to_weekday1.at("friday"))
              << std::endl;
    // Calls free() at least 7 times
  }
  {
    // Calls malloc() at least 8 times
    static const std::unordered_map<string_view, weekday> to_weekday2{
      std::begin(string_to_weekday), std::end(string_to_weekday)};
    std::cout << "'monday' maps to "
              << static_cast<int>(to_weekday2.at("monday"))
              << std::endl;
    std::cout << "'friday' maps to "
              << static_cast<int>(to_weekday2.at("friday"))
              << std::endl;
    // Calls free() at least 8 times
  }
  return 0;
}
\end{lstlisting}
\end{spacing}

    \noindent We have some data available to us at compile-time and we would
    like to perform lookups at both compile-time and run-time. Using
    \texttt{std::unordered\_map} or \texttt{std::map} we get $\mathcal{O}(1)$ or
    $\mathcal{O}(\log N)$ lookups respectively, but at the cost of
    $\mathcal{O}(N)$ memory allocations. On top of that, even when keys we are
    looking for are known at compile-time, the lookups themselves will only
    happen at run-time. Consider now the very same example written in terms of
    \texttt{static\_views::static\_map}:
\begin{spacing}{0.8}
\begin{lstlisting}
static auto to_weekday = static_views::make_static_map(string_to_weekday);
std::cout << "'monday' maps to "
          << static_cast<int>(to_weekday1.at("monday"))
          << std::endl;
std::cout << "'friday' maps to "
          << static_cast<int>(to_weekday1.at("friday"))
          << std::endl;
\end{lstlisting}
\end{spacing}
    \noindent Firstly, we get rid of memory allocations. Secondly, recent
    compilers are able to perform the lookups at compile-time. And finally,
    seeing as we have declared \texttt{string\_to\_weekday} as
    \texttt{constexpr}, we can make \texttt{to\_weekday} \texttt{constexpr} as
    well which will eliminate the initialisation costs.

    \paragraph{Better switch} One can also use \texttt{static\_map} as a
    \texttt{switch} statement:

\begin{spacing}{0.8}
\begin{lstlisting}
enum class animal : unsigned {
  cat    = 0x11,
  dog    = 0x22,
  turtle = 0x33,
  ... a couple more ...
};

// Implementation using switch
auto operator<<(std::ostream& out, animal const x) -> std::ostream&
{
  switch (x) {
  case animal::cat:    return out << "cat";
  case animal::dog:    return out << "dog";
  case animal::turtle: return out << "turtle";
  default: throw std::invalid_argument{"What has just happened?"};
  };
}

// Implementation using static_map
auto operator<<(std::ostream& out, animal const x) -> std::ostream&
{
  constexpr static std::pair<animal, char const*> mapping[] = {
    {animal::cat,    "cat"},
    {animal::dog,    "dog"},
    {animal::turtle, "turtle"}
  };
  constexpr auto animal_to_string = static_views::make_static_map(mapping);
  return out << animal_to_string.at(x);
}
\end{lstlisting}
\end{spacing}

    This very simple example also illustrates the advantages of
    \texttt{static\_map} over \texttt{switch}. Notice that \texttt{cat} and
    \texttt{dog} do not map to consecutive integers. This means that
    algorithmically, the time complexity of first implementation is linear in
    the number of animals while that of the second is (amortised) constant. This
    becomes important if we, for example, want to implement a custom
    \texttt{error\_category}. Quite often, error codes are not just consecutive
    integers and if there are many of them (e.g. there are thousands of possible
    \texttt{NTSTATUS} values) using $\mathcal{O}(1)$ implementation over
    $\mathcal{O}(N)$ one starts to make sense.

    \paragraph{What needs to be done} The goal of this project is to bring
    StaticViews library to ``review-ready'' state, i.e. a state in which Boost
    developers find the library fit to be put into library review queue. It is
    close to very difficult specify exactly needs to be done, be here are the
    most important points:
    \begin{itemize}
        \item Ensure that the above examples compile and run on at least two
            major compilers (\texttt{Clang} and \texttt{MSVC}) and are
            thoroughly worked out in the tutorial section of the documentation.
        \item Complete the reference part of the documentation.
        \item Polish the implementation. In particular, the current messy way of
            specifying concepts should be improved.
        \item Add automatic code coverage checks and make sure the results are
            above 90\%.
        \item Prove, by means of extensive benchmarking, that
            \texttt{static\_views::static\_map} outperforms
            \texttt{std::unordered\_map} in both the speed of initialisation and
            the speed of lookups.
    \end{itemize}
    Also, here are a few things which need to be done but are optional rather
    than essential part of the project:
    \begin{itemize}
        \item Fix \texttt{GCC} so that it can compile StaticViews.
        \item Add \texttt{CMake} support.
        \item Add more data structures such as a flap sorted map.
    \end{itemize}

\newpage
\section*{Schedule}
    I propose the following schedule to achieve the goals outlined above:

    \begin{center}
    \begin{tabular}{|r c l|p{10cm}|}
        \hline \noalign{\smallskip}
        %
        14th May    & --- & 18th May    & \multirow{2}{10cm}{%
            Add automatic code coverage tests and start polishing the
            implementation.%
        } \\
                    &     &             & \\
        %
        \noalign{\smallskip} \hline \noalign{\smallskip}
        %
        21st May    & --- & 25th May    & \multirow{2}{10cm}{%
            Keep working on the implementation and make sure both \texttt{Clang}
            and \texttt{MSVC} are happy with it.%
        } \\
        28th May    & --- & 1st  June   & \\
        %
        \noalign{\smallskip} \hline \noalign{\smallskip}
        %
        4th  June   & --- & 8th  June   & Keep working on the implementation and
                                          tests. \\
        %
        \noalign{\smallskip} \hline \noalign{\smallskip}
        %
        11th June   & --- & 15th June   & \multirow{2}{10cm}{%
            First evaluation. It would be nice to have enough unit tests and an
            implementation we are happy with.
        } \\
                    &     &             & \\
        %
        \noalign{\smallskip} \hline \noalign{\smallskip}
        %
        18th June   & --- & 22nd June   & \multirow{4}{10cm}{%
            Work out the ``\texttt{enum}$\to$string'',
            ``8-bit$\leftrightarrow$24-bit bitmaps'', and
            ``std::error\_category'' examples. Create a PR to Niall's
            \texttt{ntkernel-error-category}\cite{ntkernel-error-category}
            library to make it work correctly with StaticViews.
        } \\
        25th June   & --- & 29th June   & \\
        2nd  July   & --- & 6th  July   & \\
                    &     &             & \\
        %
        \noalign{\smallskip} \hline \noalign{\smallskip}
        %
        9th  July   & --- & 13th July   & \multirow{2}{10cm}{%
            Start working on benchmarks and (possibly) optimising the library
            based on them.
        } \\
                    &     &             & \\
        %
        \noalign{\smallskip} \hline \noalign{\smallskip}
        %
        16th July   & --- & 20th July   & \multirow{2}{10cm}{%
            Finish the benchmarks and ask for some initial feedback on
            \texttt{boost-dev} mailing list.
        } \\
                    &     &             & \\
        %
        \noalign{\smallskip} \hline \noalign{\smallskip}
        %
        23th July   & --- & 27th July   & Have a go at fixing \texttt{GCC}. \\
        %
        \noalign{\smallskip} \hline \noalign{\smallskip}
        %
        30th July   & --- & 3rd  August & \multirow{2}{10cm}{%
            Work on other optional tasks and changes based on feedback from
            \texttt{boost-dev}.
        } \\
        6th  August & --- & 10th August & \\
        13th August & --- & 17th August & \\
        %
        \noalign{\smallskip} \hline
    \end{tabular}
    \end{center}
\newpage
\bibliography{bibliography}

\end{document}
